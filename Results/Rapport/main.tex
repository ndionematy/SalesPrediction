\documentclass{rapportECC}
\usepackage{lipsum}
\title{Rapport ECL - Template} %Titre du fichier
\usepackage{lipsum} 
\usepackage{biblatex} %Imports biblatex package
\addbibresource{bibtex.bib} %Import the bibliography file
\usepackage{appendix} % Package pour gérer les annexes
\begin{document}

%----------- Informations du rapport ---------

\titre{Titre du rapport} %Titre du fichier .pdf

\sujet{\LaTeX Approfondi} %Nom du sujet

\Encadrants{Prénom \textsc{Nom}
 \\Prénom \textsc{Nom} } %Nom de l'enseignant

\eleves{Prénom \textsc{Nom} \\
		Prénom \textsc{Nom} \\ 
		Prénom \textsc{Nom} } %Nom des élèves

%----------- Initialisation -------------------
        
\fairemarges %Afficher les marges
\fairepagedegarde %Créer la page de garde
\tabledematieres %Créer la table de matières

%------------ Corps du rapport ----------------


\section{Première section} 

\lipsum[3-4]%Effacer cette ligne et écrire le texte souhaité

\subsection{Subsection}


\lipsum[3-4] %Effacer cette ligne et écrire le texte souhaité
\subsubsection{Subsubsection}
\lipsum[1-2]
\section{Deuxième section}

\lipsum[3-5] %Effacer cette ligne et écrire le texte souhaité

%------------- Commandes utiles ----------------

\section{Quelques commandes}

Voici quelques commandes utiles : \cite{Lamport}

%------ Pour insérer et citer une image centralisée -----

\insererfigure{logos/Logo_ECC}{3cm}{Légende de la figure}{Label de la figure}
% Le premier argument est le chemin pour la photo
% Le deuxième est la hauteur de la photo
% Le troisième la légende
% Le quatrième le label
Ici, je cite l'image \ref{fig: Label de la figure}


%------- Pour insérer et citer une équation --------------

\begin{equation} \label{eq: exemple}
\rho + \Delta = 42
\end{equation}

L'équation \ref{eq: exemple} est cité ici. 

% ------- Pour écrire des variables ----------------------

Pour écrire des variables dans le texte, il suffit de mettre le symbole \$ entre le texte souhaité comme : constante $\rho$. \cite{Companion}

\section{Conclusion et Perspectives}
\lipsum[1-2] \cite{matsumoto_tracking_2013}

\newpage


\printbibliography

% Ajout de l'annexe
\newpage
\begin{appendices}
\section{Annexe : }
Dans cette section de l'annexe, nous fournissons des détails supplémentaires.

% Ajout d'un tableau
\subsection{Tableau additionnel}
Voici un exemple de tableau :
\begin{table}[h]
    \centering
    \begin{tabular}{|c|c|}
    \hline
    Colonne 1 & Colonne 2 \\
    \hline
    Donnée A & 1 \\
    Donnée B & 2 \\
    \hline
    \end{tabular}
    \caption{Exemple de tableau}
\end{table}

% Ajout d'une image
\subsection{Graphique complémentaire}
Voici un exemple d'insertion d'une image :
\begin{figure}[h]
    \centering
    \includegraphics[width=0.5\textwidth]{logos/centrale.png}
    \caption{Légende de l'image}
\end{figure}
\end{appendices}

\end{document}
